\documentclass[12pt,a4paper]{article} 
\usepackage{multirow}
\usepackage{fullpage}
\usepackage{hyperref}
\usepackage{siunitx}

\begin{document}\title{Exploring Formula 1 Data Through Visualization\\INFO 422}
\author{Joe Stoica, Ethan Gomes, Jack Thompson, Sebastian Susanto}  					
\maketitle

\pagestyle{plain}

%----------------------------------------------------------------------
% Abstract
%----------------------------------------------------------------------

\section*{Abstract}
\hspace{20px}This project focuses on data collected from every Formula 1 racing season from 1950 up to the current  2018 season. The data set we will use comes from Ergast Developer API, which is a web service that provides historical records of the Formula One motorsport series. Our team plans to illustrate the major engineering advances in racing technology by highlighting the incredible progression of the fastest sport on Earth. Fastest lap records, pit stop times, Constructor Cup standings, and even the race tracks themselves will be analyzed to reveal a deeper understanding of why racing diehard fans worldwide remain loyal to this intense sport.

%----------------------------------------------------------------------
% Introduction
%----------------------------------------------------------------------

\section*{Introduction}
\hspace{20px} Formula 1 is constantly evolving. Each year, new technical and sporting regulations are released by F1's governing body, the FIA (3). With technology advancing at breakneck speeds, the FIA needs to ensure their drivers remain safe while also delivering faster and more intense races to their fans. Adapting these new technologies allow the drivers to push themselves (and their cars) to their limits. For example, in the 1950 Monaco Grand Prix, Juan Manuel Fangio had the fastest lap with a time of 1:51 on the 1.977 mile track, which is an average speed of about 64 miles per hour. To compare to this year's race, Kimi R�ikk�nen had the fastest lap with 1:06.957, at a blistering pace of 111.5 miles per hour (5)! Significant differences like this exist all throughout this data set, and it is this project's goal to illuminate these changes and tell the story of how the sport has advanced to where it is today. 

\hspace{20px} Not only are the technological developments of F1 interesting, but the history surrounding the Constructor's Cup is a theatrical experience in itself. F1 has seen some of the best drivers who have ever lived, with icons such as Michael Schumacher, Lewis Hamilton, Ayrton Senna, and Nico Rosberg. It is a heated debate on who the best driver of all time is, but many of these drivers never raced in similar conditions. It is impossible to say how Senna would race today, when the cars today are insanely different from those of 25 years ago. Therefore, quantifying these arguments is fairly difficult. Hopefully, this data can offer clarity on this discussion, and provide insight on how the greats compare to one another.

%----------------------------------------------------------------------
% Background
%----------------------------------------------------------------------

\section*{Background}
\hspace{20px} There are existing visualizations of Formula One data that have inspired us. For example, a data visualization that can be found on Tableau's public portal called "A History of F1 World Champions," which visualizes the cumulative point standings for every Driver for each season through small multiple line plots (4). Another Tableau dashboard by the user napoking provides some more insight. There are quite a few visualizations, ranging from race finishing positions to season points. However it is organized quite haphazardly, making navigation difficult and confusing (2). Similarly, there is a lovely infographic hosted on Behance that focuses on a few different F1 topics. The first infographic shows how lap times at Monaco have changed over the past few decades. We wish to expand on this idea by taking a similar approach to the data and how it is exhibited, but perhaps split the years up into more categories and examine multiple race tracks. (1) 
\hspace{20px}

\hspace{20px} We envision that our final product will we displayed on a webpage to make for easy scrolling and interacting with all of the visualizations. Instead of having one infographic, multiple visualizations will be displayed with prose describing them. One idea we have is to create an interactive world map with each track location marked, and when a location is selected, it can show the track map as well as displaying various statistics about the track, such as distance, lap records, and so on. 

%----------------------------------------------------------------------
% Research Questions
%----------------------------------------------------------------------

\section*{Research questions}
\hspace{20px}At the beginning of this project, there are a few natural questions our team wishes to explore:
\begin{itemize}  
\item How have lap times gotten faster since Formula 1's inception, and how do these changes compare by track? 
\item How have pit stop times changed? We expect them to be faster, but by how much?
\item How can we quantify driver skills, and how can we determine who is the best?
\end{itemize}
\hspace{20px}These questions are subject to change, however. As we dive deeper into the data, new findings can lead to new questions, or perhaps one of our current primary questions turns out to be a dead end.

%----------------------------------------------------------------------
% Process
%----------------------------------------------------------------------

\section*{Process}
\hspace{20px} 
\hspace{20px}

%----------------------------------------------------------------------
% Results and Insights
%----------------------------------------------------------------------

\section*{Results and Insights}
\hspace{20px} 
\hspace{20px}

%----------------------------------------------------------------------
% Discussion, Conclusion, and Future Work
%----------------------------------------------------------------------

\section*{Discussion, Conclusion, and Future Work}
\hspace{20px} 
\hspace{20px}



%----------------------------------------------------------------------
% Bibliography
%----------------------------------------------------------------------

\begin{thebibliography}{1}

  \bibitem{notes} Behance. "Formula One Infographics." {\em Behance,}  https://bit.ly/2pldARN

  \bibitem{napo} napoking. "Formula 1." {\em Tableau Public}, https://tabsoft.co/2De2Zlk
  
  \bibitem{thing} "Regulations." {\em Federation Internationale De L'Automobile} https://bit.ly/2xyYXOd

  \bibitem{impj}  Smith, James. "F1 World Champions." {\em Tableau Public,} https://tabsoft.co/2xpOXrp
  
  \bibitem{stuffff} "Standings." {\em Formula1.com} https://f1.com/2poFJHs
  

  

\end{thebibliography}

%3: 
	
\end{document}  